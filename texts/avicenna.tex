\documentclass[a4paper]{article}
% generated by Docutils <http://docutils.sourceforge.net/>
\usepackage{fixltx2e} % LaTeX patches, \textsubscript
\usepackage{cmap} % fix search and cut-and-paste in Acrobat
\usepackage{ifthen}
\usepackage[T1]{fontenc}
\usepackage[utf8]{inputenc}

%%% Custom LaTeX preamble
% PDF Standard Fonts
\usepackage{mathptmx} % Times
\usepackage[scaled=.90]{helvet}
\usepackage{courier}

%%% User specified packages and stylesheets
%% Default local configuration (simple LaTex stylesheet) for rst2latex.
%% call
%% rst2latex --stylesheet=rst2latex.sty (input).txt (output).tex

\usepackage{parskip}


%%% Fallback definitions for Docutils-specific commands

% hyperlinks:
\ifthenelse{\isundefined{\hypersetup}}{
  \usepackage[colorlinks=true,linkcolor=blue,urlcolor=blue]{hyperref}
  \urlstyle{same} % normal text font (alternatives: tt, rm, sf)
}{}
\hypersetup{
  pdftitle={Bewerbung um ein Stipendium bei Avicenna Studienwerk e.V.},
}

%%% Title Data
\title{\phantomsection%
  Bewerbung um ein Stipendium bei Avicenna Studienwerk e.V.%
  \label{bewerbung-um-ein-stipendium-bei-avicenna-studienwerk-e-v}%
  \\ % subtitle%
  \large{(Motivationsschreiben)}%
  \label{motivationsschreiben}}
\author{}
\date{}

%%% Body
\begin{document}
\maketitle

% @file: avicenna.txt

% @syntax: reStructuredTex

% @date: 24 Mär 2016

% @author: Aroldo Souza-Leite

Hadith: \emph{Du sollst von der Wiege bis zur Bahre lernen!}

Schon die Namensgebung für dieses Studienwerk, Avicenna, ist zu meiner persönlichen Motivation, mich intellektuell und fachlich immer weiter zu entwickeln, sehr kompatibel. Avicenna ist meines Erachtens ein Symbol für die im Westen nicht seltenen Unterschätzung des Einflusses des Ostens auf das hiesige wissenschaftlich-kulturelle Erbe und der hohen Bewertung von Lernen im islamischen Kulturraum.

Als in Deutschland geborener und eingeschulter Türke islamischen Glaubens war mein Alltag, auch ohne dass ich mich dessen dauernd bewusst war, immer von kulturellen Strömungen zwischen diesen zwei Welten geprägt. Erst die Erhöhung meines Bildungsniveaus macht es mir möglich, diese Strömungen klar zu orten und die Kenntnis darüber zum Vorteil meines sozialen Umfelds auf beiden Seiten dieser Wechselwirkung einzusetzen.

Mein Großvater mütterlicherseits war in Deutschland ein, wie es früher hieß, Gastarbeiter. Er stammt aus einem ländlichen Gebiet der Türkei. Meine Eltern sind noch im Jugendalter aus dem gleichen Gebiet zu ihm nachgezogen. Sie haben einen geringen schulischen Hintergrund mitgebracht und fanden hier in Deutschland, bedingt durch ihre Arbeitsverhältnisse, keine Gelegenheit, sich weiter zu bilden.

Selbst bin ich in Köln geboren und eingeschult. Ich habe einen älteren Bruder und eine jüngere Schwester, die ebenfalls in Deutschland zur Schule gingen und hier leben.

Als Kind und als Jugendlicher lebte ich mit der fünfköpfigen Familie meiner Eltern in nach deutschen Standards recht engen räumlichen Verhältnissen. Ich teilte ein Zimmer mit meinem Bruder und später mit meiner jüngeren Schwester. Entsprechend schwer war es für uns Kinder Lernzeiten und -möglichkeiten zu organisieren. Wir sind aber von unseren Eltern dazu erzogen worden, zu teilen. Wir haben auf dieser Grundlage die Schwierigkeiten gut bewältigen können. Ich nehme diese Erfahrung der solidarischen Teilung von geringen Ressourcen als Bereicherung meines Charakters fürs Leben mit. Ich genieße heute als Erwachsener die Achtung und die Liebe meiner Geschwister.

Es gehörte zu den Selbstverständlichkeiten für mich als Jugendlichen, dass mein älterer Bruder mich als Gehilfe in seinem damaligen Imbissladen in der Kölner Innenstadt oft einsetzte. Auch meinem Vater half ich oft in seinem Nebenjob als Gebäudereinigungskraft. Dabei  habe ich sehr früh Erfahrung mit geschäftlichen Vorgängen gesammelt und gleichzeitig eine hohe Achtung für nicht akademische Arbeit, insbesondere von Migranten, verinnerlicht.

Meine Eltern haben immer Wert darauf gelegt, dass wir schon als Kinder parallel zur deutschen Schule eine solide religiös geprägte Erziehung erhalten. Daher sind wir jedes Wochenende zu den von den Moscheen veranstalteten Unterrichtsstunden gegangen. Dort haben wir Koran lesen gelernt und in diesem Umfeld unsere Allgemeinbildung vertieft. In der Koran-Schule habe ich den Islam als wissenschafts- und bildungsfreundliche Religion erfahren. Mein heutiges Streben nach Weiterbildung wurzelt in dieser Zeit.

Dieses Umfeld hat mich dazu motiviert, mich sozial zu engagieren. So habe ich interkulturelle Veranstaltungen mitgestaltet und -organisiert, Nachhilfe bei Schulaufgaben an jüngere Menschen geleistet und sonstige soziale Arbeiten verrichtet.

Vor meiner Ausbildung habe ich als Hauptschulabsolvent in verschiedenen Branchen gearbeitet, unter anderem bei UPS als Lagergehilfe, Dokumentenprüfer und als  Standort- und Firmenrecherchenarbeiter für Kuriere.

In meinem Ausbildungsbetrieb, der Kara AG, wurde meine Fähigkeit, selbstorganisiert zu arbeiten, stark in Anspruch genommen. Ich war in der Firma oft de facto als leitender Mitarbeiter tätig. Dadurch bin ich auch als Auszubildender zur Selbständigkeit erzogen worden.

Das Georg-Simon-Ohm-Berufskolleg (GSO), meine Ausbildungsschule, ist im Bereich der Medientechnik und der IT eins der bekanntesten Berufskollegs der Region. Dort habe ich nicht nur eine technische Allgemeinbildung erhalten, sondern auch unter einem engagierten Deutsch- und Wirtschaftslehrer meine Sprachkenntnisse erheblich verbessert und generell eine Erziehung zum europäischen Bürger genossen.

In meiner Stufe waren unter insgesamt 23 Schülern nur zwei mit Migrationshintergrund. Ich war der älteste der Stufe und einer der sehr wenigen, die mit betrieblicher Arbeitserfahrung in die Ausbildung eingestiegen ist. Schon deswegen war die Inanspruchnahme meiner Arbeitskraft in meinen Ausbildungsbetrieb intensiver und belastender als die meiner Mitschüler. Das habe ich als Ansporn genutzt, so viel wie nur möglich aus meiner Ausbildungsschule herauszunehmen und war unter anderem Klassensprecher.

Ich war Austauschschüler des GSO beim Leonardo-da-Vinci-Projekt in Spanien und bin heute Besitzer eines Zeugnisses der Europass Mobility. Meinen Spanien-Aufenthalt habe ich auch genutzt, um mich mit der europäischen und der arabischen Kultur näher zu befassen. Mein Austauschbetrieb lag in Malaga, der Geburtsstadt Pablo Picassos, wo ich das ihm gewidmete Museum besichtigte. In Granada habe ich an der Alhambra die maurische Architektur von der Nähe erlebt. Mein Spanien-Aufenthalt war ein wichtiger Beitrag zur Reflektion über mein Leben zwischen verschiedenen Kulturen.

Zusätzlich zum Gesellenbrief als Fachinformatiker der Industrie- und Handelskammer erhielt ich vom   Georg-Simon-Ohm-Berufskolleg aufgrund meiner Gesamtnote und meiner Englischkenntnisse das Zeugnis der Fachoberschulreife.

Mit dieser Qualifikation bin ich vom Werner-von-Siemens-Berufskolleg (WvS) aufgenommen worden. Dort habe ich mich weiter gebildet bis zum Erlangen der Fachhochschulreife mit Schwerpunkt Elektrotechnik.

Im WvS bin ich zum ersten Mal in Kontakt mit höherer Mathematik (Differential- und Integralrechnung, Fourieranalysis, komplexen Zahlen) gekommen. Das hat mir Spaß gemacht und ich freue mich auf eine weitere Beschäftigung mit mathematisch geprägten Themen im meinem Studium.

Das WvS hat zugesehen, dass, auch wenn mein Schwerpunkt Elektrotechnik war, meine Allgemeinbildung erweitert wurde. Vor allem war der Englischunterricht didaktisch sehr gekonnt gestaltet. Im Deutschunterricht haben wir unter anderem Dürrenmatt gelesen und eine Theateraufführung von \textquotedbl{}Die Physiker\textquotedbl{} gemeinsam besucht.

Ich habe eine sechs Monate alte Tochter und bin ein aktiv miterziehender Vater. Dies nimmt mich sehr stark in Anspruch, ist aber gleichzeitig für mich eine Quelle von Lebensfreude und Motivation für mein Studium und für die Bewältigung der Hürden, die damit verbunden sind.

Trotz der geringen Freizeit nehme ich weiterhin Teil an Veranstaltungen im Umfeld der Moschee. Dies schließt ein sowohl die Teilnahme an Veranstaltungen für meine eigene Weiterbildung als auch die Beteiligung an Spendensammlungen für soziale Zwecke, wie z. B. die Förderung von jüngeren Religionsstudenten.

Ich bin sportlich aktiv und im Rahmen meiner zeitlichen Möglichkeiten organisiere ich einen Sportverein in meinem Stadtteil mit.

Es ist meine feste Absicht, mein Studium so zu gestalten, dass ich im Anschluss dazu ein Masters of Science und später ein Ph.D. machen kann. In diesem Zusammenhang ist es für mich günstig, dass meine Hochschule jüngst von einer Fachhochschule zu einer Technischen Hochschule, also einer Universität mit Promotionsrecht, aufgestiegen ist.

Aktuell strebe ich gute Noten zum Abschluss des zweiten Semesters an, um im darauffolgenden Semester als Mentor an der Hochschule tätig zu werden. Generell möchte ich von vorne herein, schon als Bachelor-Student, die Entwicklung der neuen TH von innen verfolgen -  unter anderem indem ich mich früh mit Forschungsvorhaben und wissenschaftlichen Publikationen der Professoren bekannt mache.

Mir schwebt vor, nach dem Studium als Wirtschaftsberater mit einer soliden IT-Grundlage beruflich tätig zu werden.

Im Rahmen des ideellen Programms des Avicenna-Studienwerks möchte ich mich mit folgenden drei Themen auseinandersetzen:
%
\begin{quote}
%
\begin{itemize}

\item Zugänglichmachen von neuen digitalen Techniken für Menschen aller Schichten
und sozialen Gruppen, z.B. muslimische Frauen.

\item Integration von Migranten unter dem Vorzeichen der Pflege und Verstärkung
der sprachlichen und kulturellen Beziehung zum Herkunftsland

\item Chancen und Herausforderungen des interreligiösen Dialogs

\end{itemize}

\end{quote}

\end{document}
